%%---------------------------------------------------------------------------%%
\section{Conclusions and Future Work}
\label{sec:conc}

In this report we discussed a series of simulations performed with NekRS for an SMR full core simulation. We reiterate that \textit{this milestone provides the first full-core pin resolved simulations ever performed to our knowledge}. The capability developed is significant advancement for the field of applied CFD in nuclear engineering.

The simulations perfortmed covered both Large Eddy Simulation and Reynolds Averaged Navier Stokes. The Reynolds Averaged Navier-Stokes simulations include modeling of the spacer grid effects, a key aspect of the flow in PWR cores, through momentum sources. The momentum sources have been developed and calibrated using Large Eddy Simulation in a 5x5 assembly with the spacers and the vanes explicitly modeled. We note that there is considerable literature already on the validation of Nek5000 for spacer grids \cite{BUSCO2019144} \cite{yuan2020spectral}. This report presents the first set of simulations in which a momentum sources approach has been applied to a full core.

The Large Eddy Simulation results for the full core have been used to define a new FOM for ExaSMR. The obtained results mark a significant improvement compared to the previous value on Summit (4.8x increase). They also represent the first full machine measurement with NekRS.  The performance is now two orders of magnitude higher than on Titan (with the OpenACC version of Nek5000). We note that this corresponds to broad improvements to the code and it is not an isolated case. Similar speed-ups have been reported for a broad range of production runs.

In the future we will perform full core, full height conjugate heat transfer calculations and work toward performing full core coupled calculations with Monte Carlo. This is currently limited by integer overflow in Nek5000 and NekRS (the maximum number of elements is $E=180,000,000$). We are working to remove this limitation. We will also implement a zonal hybrid approach  for the full core. In this approach the core is modeled with URANS and momentum source while in one of the assemblies the spacer grids are modeled explicitly and turbulence is modeled with LES.
